\section[Generalities on algebraic extensions]{Lecture Notes: 22 Feb -- 28 Feb}
\subsection{Field extensions. Examples.}

This course assumes a basic knowledge of abstract algebra (groups, rings, fields, modules), and linear algebra. All rings we consider will be associative, commutative, and with unity.

\subsubsection{Two definitions of field extension.}

Let $K$ and $L$ be fields.

\begin{defn}
We say that $L$ is an \term{extension of $K$} if $K \sbs L$. That is, $K$ is a subfield of $L$. Equivalently, $L$ is an extension of $K$ if $L$ is a \term{$K$-algebra}---in other words, if we have $(k_1 \mathbf{a_1})(k_2 \mathbf{a_2}) = k_1 k_2 \mathbf{a_1} \mathbf{a_2}$ for $k_i \in K$ and $\mathbf{a_i} \in A$.
\end{defn}

Why are these definitions equivalent? In fact, given a $K$-algebra structure on a ring $A$, this is the same as having a homomorphism of rings $f: K \to A$. So if we have a $K$-algebra, define a homomorphism $f$ by setting $f(k) = k \mathbf{1}$ for $k \in K$. Conversely, given an arbitrary homomorphism $f: K \to A$, set $k\mathbf{a} = f(k)\mathbf{a}$ for $\mathbf{a} \in A$.

Suppose now that $A = L$ a field. Then any homomorphism $f: K \to L$ is injective. There are several ways to see this; for example, we can show that $f(k)$ is always invertible. Indeed, $\mathbf{1} = f(1) = f(k k^{-1}) = f(k)f(k^{-1})$ for any $k \neq 0$, so $f(k) \neq \mathbf{0}$ whenever $k$ is nonzero. Alternatively, we know that the kernel of $f$ is always an ideal. But $L$ is a field, so the only ideals of $L$ are $(0)$ and $(1) = K$.

\subsubsection{Three examples.} 
\begin{ex}
$\C$ is an extension of $\R$, and $\R$ is an extension of $\Q$.
\end{ex}

\begin{ex}
If $L$ is a field, then either (a) $1 + 1 + \ldots + 1 \neq 0$ for any sum of $1$'s. Then $L$ has characteristic $0$ and so we have $\Z \sbs L$, which means $\Q \sbs L$. Then $L$ is an extension of $\Q$. Alternatively, suppose (b) $1 + 1 + \dotsb + 1 = 0$ for some finite $m$ number of terms. The minimal such number for which this is true turns out to necessarily be a prime, $p$. We then say that $L$ has characteristic $p$, and so we have $\Z / p\Z \sbs L$; $\Z / p\Z$ is a field, and we denote it (with field structure) by $\F_p$. In this case $L$ is an extension of $\F_p$. We call $\Q$ and $\F_p$ the \term{prime fields}: any field is an extension of a prime field, and prime fields don't contain any proper subfields.
\end{ex}

\begin{ex}
Take $K[x]/(P)$, the ring of polynomials in one variable over $K$, modded out by the ideal of an irreducible polynomial $P$. This is a field. Suppose $Q \notin (P)$, then $\gcd(Q, P) = 1$, so for some polynomials $A, B$ we have $AP + BQ = 1$ by B\'{e}zout's identity. Hence $BQ \equiv 1 \pmod P$, that is, $B$ is an inverse of $Q$ in $K[x]/(P)$.
\end{ex}

\subsection{Algebraic elements. Minimal polynomial.}
We continue with the previous example: the quotient $K[x]/(P)$ is a field. Rather than B\'{e}zout's identity, we can say that $(P)$ is a \term{maximal ideal} of $K[x]$, and the quotient of a ring by a maximal ideal is always a field. The proof of this fact uses the same identity.

This field is an extension of $K$ in the obvious way: it is a $K$-algebra!

\subsubsection{A concrete example.} 
Let $K = \F_2 = \{0, 1\} = \Z / 2\Z$, and $P = x^2 + x + 1$. Then $K[x] / (P)$ contains four elements: $0$, $1$, the class containing $x$ (denoted by $\bar{x}$, and the class containing $x + 1$ (denoted by $\overline{x + 1}$). We have that $\bar{x}^2 = -\bar{x} - 1 = \overline{x + 1}$ since $K$ has characteristic $2$. Similarly $(\overline{x + 1})^2 = \bar{x}$. Moreover, these elements are inverses of each other: $\bar{x}(\overline{x + 1}) = \bar{x}^2 + \bar{x} = -1 = 1$. Since $|K[x] / (P)| = 4$, we write $K[x] / (P) = \F_4$. This notation seems presumptuous, implying that there is ``only" one field with four elements: in fact every field with a given finite number of elements is isomorphic, so this is true. A proof will come later.

\subsubsection{Algebraic elements of a field extension.}
\begin{ex}
Given a field extension $K \sbs L$ and an element $\alpha \in L$, we say that $\alpha$ is \term{algebraic} if there exists some polynomial $P \in K[x]$ such that $P(\alpha) = 0$; if no such polynomial exists, we say that $\alpha$ is \term{transcendental}.
\end{ex}

\begin{lem}
If $\alpha$ is algebraic, then there exists a \emph{unique} unitary polynomial $P$ of minimal degree with $P(\alpha) = 0$. $P$ is irreducible, and for any $Q$ such that $Q(\alpha) = 0$, then $Q$ is divisible by $P$.
\end{lem}

\begin{defn}
We call such a polynomial $P$ the \term{minimal polynomial of $\alpha$ over K}, denoted $P_{\min}(\alpha, K)$.
\end{defn}
\begin{proof}
[Proof of lemma] We know that $K[x]$ is a \term{principal ideal domain}, and the polynomials $I = \{Q \in K[x] : Q(\alpha) = 0$ forms an ideal. Thus $I$ has a generator, so $I = (P)$ for some $P$. This generator is a unique (up to a constant) element of minimal degree in $I$. Furthermore, if $P$ was \emph{not} irreducible---if $P = QR$---then $P(\alpha) = Q(\alpha) R(\alpha)$ and so at least one of $Q(\alpha) = 0$ or $R(\alpha) = 0$. This would contradict the minimal-degree condition on $P$.
\end{proof}

\subsection{Algebraic elements. Algebraic extensions.}
\subsubsection{An important bit of notation.}
\begin{defn}
We denote by $K(\alpha)$ the smallest subfield of $L$ containing $\alpha$. We say that $K[\alpha]$ (note the square braces) is the smallest subring (or $K$-algebra) containing $K$ and $\alpha$.
\end{defn}

$K[\alpha]$ is generated, as a vector space over $K$, by $1, \alpha, \alpha^2, \dotsc, \alpha^n, \ldots$.

\begin{ex}
$\C = \R(\imath)$ as a field, but also $\C = \R[\imath]$ as a ring. Every $z \in \C$ can be written $z = x + \imath y$; this is a vector subspace generated by $1, \imath$.
\end{ex}

\begin{prop}
The following are equivalent: (1) $\alpha$ is algebraic over $K$; (2) $K[\alpha]$ is a finite dimensional vector space over $K$; (3) $K[\alpha] = K(\alpha)$.
\end{prop}
\begin{proof}
$(1) \implies (2)$: We have that $\alpha^{d} + a_{d-1} \alpha^{d-1} + \ldots + \alpha_{1}\alpha + a_{0} = 0$ for $a_i \in K$ (this is just the minimal polynomial). Then $\alpha^{d} = - \left(\sum_{k = 0}^{d-1} a_{k} \alpha^{k}\right)$, a linear combination of the lower powers of $\alpha$. Therefore $K[\alpha]$ is generated by $1, \alpha, \ldots, \alpha^{d-1}$ over $K$: it is finite-dimensional.

$(2) \implies (3)$: It is enough to prove that $K[\alpha]$ is a field, since $K[\alpha] \sbs K(\alpha)$. Let $x \in K[\alpha]$ nonzero. We want to show that $x$ is invertible. Consider the operation of multiplication by $x$, that is, $y \mapsto xy$ for $y \in K[\alpha]$: this is an injective homomorphism of vector spaces over $K$. But as $K[\alpha]$ is finite-dimensional, this is also a surjection, so there exists $z \in K[\alpha]$ such that $xz = 1$. Hence $x$ is invertible, and so $K[\alpha]$ is a field.

$(3) \implies (1)$: If $\alpha$ is not algebraic, then there exists no polynomial $P$ such that $P(\alpha) = 0$. This means that the natural homomorphism $i: K[x] \to L$ defined by $P \mapsto P(\alpha)$ is injective, but $K[\alpha]$ is \emph{not} a field, and the image of $i$ is a field. Contradiction!
\end{proof}

\subsubsection{Definition and properties of algebraic extensions.}

\begin{defn}
$L$ is called \term{algebraic} over $K$ if every element of $L$ is algebraic over $K$.
\end{defn}

\begin{prop} 
If $L$ is algebraic over $K$, then any $K$-subalgebra of $L$ is a field.
\end{prop}
\begin{proof}
Let $L^\prime \sbs L$ be a subalgebra. We know that $\alpha \in L^\prime$ algebraic. Then $K[\alpha] \sbs L$ is a field, so $\alpha$ is invertible (when nonzero). This holds for any such (nonzero) $\alpha$, so $L^\prime$ is a field.
\end{proof}

\begin{prop}
If $K \sbs L \sbs M$, and $\alpha \in M$ is algebraic over $K$, then $\alpha$ is algebraic over $L$ and its minimal polynomial $P_{\min}(\alpha, L)$ divides $P_{\min}(\alpha, K)$.
\begin{proof}
Consider $P_{\min}(\alpha, K)$ as an element of $L[x]$.
\end{proof}
\end{prop}

\subsection{Finite extensions. Algebraicity and finiteness.}
\begin{defn}[Finite extension]
$L$ is said to be a \term{finite extension} of $K$ if it is a finite-dimensional $K$-vector space. The dimension of $L$ over $K$ is called the \term{degree} of $L$ over $K$, and is denoted by $[L:K]$. 
\end{defn}
\begin{thm}
Suppose $K \sbs L \sbs M$. Then $M$ is finite over $K$ if and only if $M$ is finite over $L$ and $L$ is finite over $K$. Moreover, in this case, the degrees multiply: $[M:K] = [M:L][L:K]$.
\end{thm}
\begin{proof}[Proof of Thm. 1]
First, suppose $M$ is finite over $K$. Then any linearly independent family $\{m_i\}$ over $L$ are also linearly independent over $K$, so $\dim_L M$ is finite. Now $L$ is a $K$-vector subspace of $M$, so $\dim_K M$ is finite and thus $\dim_K L$ is finite.

Second, let $\set{e_i}_{i = 1}^{n}$ be an $L$-basis of $M$, and $\set{\epsilon_j}_{j = 1}^{d}$ a $K$-basis of $L$. We want to show that $e_i \epsilon_j$ form a $K$-basis of $M$. Indeed, for any $x \in M$, we have that $x = \sum_{i} a_i e_i$ with $a_i \in L$. And for each $i$, $a_i = \sum_{j} b_{ij} \epsilon_j$ with $\sum_{i,j} b_{ij}\epsilon_j \in K$. So we can write $x = \sum_{i,j} b_{ij} \epsilon_j e_i$, showing that $e_i \epsilon_j$ generate $M$ over $K$. We now need to verify that these elements are linearly independent over $K$.

If we have $\sum_{i,j} c_{ij} e_i \epsilon_j = 0$ then $\sum_{i} \left(\sum_{j} c_{ij} \epsilon_j\right) e_i = 0$, and $\sum_{j} c_{ij} \epsilon_j \in L$. But $\set{e_i}$ is a basis, so for all $i$, we have $\sum_{j} c_{ij} \epsilon_j = 0$. And since $\set{\epsilon_j}$ is a basis, necessarily $c_{ij} = 0$ for all $i, j$. This proves the theorem.
\end{proof}

\begin{defn}
We say that $K(\alpha_1, \dotsc, \alpha_n) \sbs L$, the smallest subfield of $L$ containing $K, \alpha_1, \dotsc, \alpha_n$, is \term{generated} by $\alpha_1, \dotsc, \alpha_n$ over $K$.
\end{defn}

\begin{thm}
$L$ is finite over $K$ if and only if $L$ is generated by a finite number of algebraic elements over $K$.
\end{thm}
\begin{proof}
First, suppose that $\set{\alpha_i}_{i=1}^{d}$ is a $K$-basis of $L$. Then $L = K[\alpha_1, \dotsc, \alpha_d] = K(\alpha_1, \dotsc, \alpha_d)$. Moreover, each $K[\alpha_i]$ is a finite-dimensional $K$-algebra since it is a subring of (already finite-dimensional) $L$. Then by Proposition 1, $\alpha_i$ is algebraic.

Second, suppose $K[\alpha_1]$ is finite dimensional over $K$; $K[\alpha_1, \alpha_2]$ is finite dimensional over $K[\alpha_1]$; \ldots; $K[\alpha_1, \dotsc, \alpha_{d-1}, \alpha_d]$ finite dimensional over $K[\alpha_1, \dotsc, \alpha_{d-1}]$. Each $\alpha_i$ is algebraic, so for $1 \leq i \leq d$ we have $K[\alpha_1, \dotsc, \alpha_i] = K(\alpha_1, \dotsc, \alpha_i)$. Now we use Theorem 1 to conclude that $L = K(\alpha_1, \dotsc, \alpha_d)$ is finite over $K$.
\end{proof}

\subsection{Algebraicity in towers. An example.}
Algebraic extensions have a similar property to finite extensions: a tower of extensions is algebraic only if the floor of the tower is algebraic.

\begin{thm}
Let $K \sbs L \sbs M$. Then $M$ is algebraic over $K$ if and only if $M$ is algebraic over $L$ and $L$ is algebraic over $K$.
\end{thm}
\begin{proof}
First, let $\alpha \in M$. If $P(\alpha) = 0$ for some $P \in K[x]$, then also $P \in L[x]$, so $\alpha$ is algebraic over $L$. Now if $\alpha \in L$ then also $\alpha \in M$ and so $\alpha$ is algebraic over $K$. Thus $L$ is algebraic over $K$.

Second, suppose $L$ is algebraic over $K$ and $M$ is algebraic over $L$; we need to show that $M$ is algebraic over $K$. Take $\alpha \in M$ and consider $P_{\min}(\alpha, L)$. Its coefficients are elements of $L$, so they are algebraic over $K$. By the previous theorem, they generate an extension, $E$, which is \emph{finite} over $K$. Now $E(\alpha)$ is also finite over $K$. Since $E(\alpha)$ is finite over $E$, then $\alpha$ is algebraic over $K$: there exists a linear dependence relation between powers of $\alpha$.
\end{proof}

We now consider an example.

\begin{ex}
Consider $\Q(\sqrt[3]{2}, \sqrt{3})$. This is clearly algebraic and finite over $\Q$. The degree of this extension is 6: we have $\Q \sbs \Q(\sqrt[3]{2}, \sqrt{3})$. The minimal polynomial $P_{\min}(\sqrt[3]{2}, \Q) = x^3 - 2$; $\Q(\sqrt[3]{2})$ is generated over $\Q$ by $1, \sqrt[3]{2}, (\sqrt[3]{2})^2$, so $[\Q(\sqrt[3]{2}) : \Q] = 3$.

Now $\sqrt{3} \notin \Q(\sqrt[3]{2})$, because otherwise we would have $\Q \sbs \Q(\sqrt{3}) \sbs \Q(\sqrt[3]{2})$. Then $2 = [\Q(\sqrt{3}):\Q]$ would divide $3 = [\Q(\sqrt[3]{2}):\Q]$, which is impossible. Therefore, $x^2 - 3$ is irreducible over $\Q(\sqrt[3]{2})$, and so is in fact the minimal polynomial for $\sqrt{3}$ over this extension.

The degree of the big extension, $[\Q(\sqrt[3]{2}, \sqrt{3}):\Q(\sqrt[3]{2})] = 2$, and therefore $[\Q(\sqrt[3]{2}, \sqrt{3}):\Q] = (2)(3) = 6$.
\end{ex}
In fact, this reflects a more general property:
\begin{prop}
If $\alpha$ is algebraic over $K$, then the degree of $K(\alpha)$ over $K$ is equal to the degree of the minimal polynomial of $\alpha$ over $K$.
\end{prop}
\begin{proof}
The proof is obvious: $K(\alpha)$ is generated by the powers of $\alpha$ up to some $\alpha^{d-1}$ (if $\deg P_{\min}(\alpha, K) = d$), and these are linearly independent.
\end{proof}

This gives us a nice tool to compute the degree of algebraic extensions.

\begin{prop}
Let $K \sbs L$ be a field extension and let $L^\prime = \set{\alpha \in L : \text{$\alpha$ is algebraic over $K$}}$. Then $L^\prime$ is a subfield of $L$; we call this the \term{algebraic closure} of $K$ in $L$.
\end{prop}
\begin{proof}
Let $\alpha, \beta$ be algebraic over $K$. We want to show that $\alpha + \beta$ and $\alpha\beta$ are algebraic; these facts follow immediately from Theorem 2, since $\alpha + \beta$ and $\alpha\beta$ belong to $K[\alpha, \beta]$, which is a finite (by Theorem 2) extension of $K$.
\end{proof}

\subsection{A digression: Gauss lemma, Eisenstein criterion.}
\subsubsection{A brief review.}
We said that for a field $K$, an element $\alpha$ is algebraic over $K$ if $\alpha$ is a root of some polynomial $P \in K[x]$. 

We said that an extension $L$ is algebraic over $K$ if every element $\alpha \in L$ is algebraic over $K$. 

We said that $L$ is finite over $K$ if the dimension of $L$ over $K$ is finite.

We saw that finite implies algebraic, and that we have finiteness if and only if the field is algebraic \emph{and} finitely generated.

Finally, we deduced that $[K(\alpha):K] = \deg P_{\min}(\alpha, K)$.

Therefore, it's important to be able to know whether a given polynomial is in fact irreducible over $K$.

\subsubsection{How to decide that a polynomial is irreducible over K.}
In our example we had $x^3 - 2$ is irreducible $\Q$. Since the degree of this polynomial is equal to 3 and there is no root in $\Q$.

But if we ask whether $x^{100} - 2$ is irreducible over $\Q$, this is not so trivial. In fact it is irreducible, based on a few facts.

\begin{lem}[Gauss]
Let $P \in \Z[x]$. If $P$ decomposes nontrivially (that is, $P = QR$, where $\deg Q, \deg R < \deg P$) over $\Q$, then it also decomposes over $\Z$.
\end{lem}
\begin{proof}
Let $P = QR$. Set $m Q = Q_1 \in \Z[x]$ and $n R = R_1 \in \Z[x]$. Then $mn P = Q_1 R_1 \in \Z[x]$. For $p | mn$, then modulo $p$ we have $0 = \bar{Q}_1 \bar{R}_1$. Since we're working over $\F_p$ a field, we have that $\bar{Q}_1 = 0 \pmod p$ or $\bar{R}_1 = 0 \pmod p$: that is, $p$ divides all of the coefficients of either $Q_1$ or $R_1$. WLOG say this is $Q_1$. Then $\frac{mn}{p} P = Q_2 R_1 \in \Z[x]$ where $Q_2 = \frac{Q_1}{p}$. Continuing in this way, we arrive at $P = Q_l R_s \in \Z[x]$.
\end{proof}
\begin{ex}[Eisenstein criterion example]
To show that $x^{100} - 2$ is irreducible over $\Z$? We reduce modulo 2: if $x^{100} - 2 = QR$ then $x^{100} = \bar{Q} \bar{R}$ in $\F_2[x]$, so $\bar{Q}$ and $\bar{R}$ are of the form $x^k$ respectively $x^l$. The constant coefficients of both $\bar{Q}$ and $\bar{R}$ must be divisible by 2; hence the constant coefficient of $x^{100} - 2$ must be divisible by 4, except this is not the case. Therefore 
\end{ex}

\begin{prop}[Eistenstein criterion]
Let $P \in \Z[x]$ with $P = a_n x^n + a_{n-1} x^{n-1} + \dotsb + a_0$. If there exists a prime $p$ such that (1) $p$ divides $a_n$; (2) $p$ divides $a_i$ for $i = 0, \dotsc, n-1$; and (3) $p^2$ does not divide $a_0$; then $P \in \Z[x]$ is irreducible.
\end{prop}
\begin{proof}
The proof is the same as in the example.
\end{proof}

Both facts are valid in more generality, by replacing $\Z$ with any unique factorization domain $R$, and replacing $\Q$ by the fraction field of $R$.