\subsection[Quiz]{Week 4 Quiz}
\subsubsection*{1. Which of the following modules are non-zero?}
\paragraph*{a. $\Z / 3\Z \tensor{\Z} \Z / 5\Z$?}
We use Proposition 11 with $I = (3)$ as our ideal. Hence \[\Z / 3\Z \tensor{\Z} \Z / 5\Z \iso (\Z / 5\Z) / ((3) \cdot (\Z / 5\Z))\] But $(3) \cdot (\Z / 5\Z) = \Z / 5\Z$ since $\gcd(3, 5) = 1$. Therefore the tensor product is zero.

\paragraph*{b. $\Z / 3\Z \tensor{\Z} \Z / 9\Z$?}
Again Prop. 11 yields $\Z / 3\Z \tensor{\Z} \Z / 9\Z \iso (\Z / 9\Z) / ((3) \cdot (\Z / 9\Z))$, but this time $(3) \cdot (\Z / 9\Z) \iso \Z / 3\Z$, so the tensor product is isomorphic to $(\Z / 9\Z) / (\Z / 3\Z) \iso \Z / 3\Z$.

\paragraph*{c. $\Q[x] / (x - 1) \tensor{\Q} \Q[x] / (x + 1)$?}
Note that we cannot use Prop. 11 here, since $(x - 1)$ is not an ideal in $\Q$. The left-hand quotient ring has elements of the form $a = \sum_{i=0}^n a_i$ for $a_i \in \Q, n < \infty$: since $x - 1 \equiv 0$ we can replace all $x$'s by $1$. Similarly, the right-hand quotient ring has elements of the form $b = \sum_{j=0}^{m} (-1)^j b_j$ for $b_j \in \Q, m < \infty$: since $x + 1 \equiv 0$ we can replace all $x$'s by $-1$.

This comes out to $\Q \times \Q$, which is nonzero.

\paragraph*{d. $\Q[x] / (x - 1) \tensor{\Q[x]} \Q[x] / (x + 1)$?}
Here we \emph{can} use Prop. 11, as $(x - 1)$ is an ideal in $\Q[x]$. Hence \[\Q[x] / (x - 1) \tensor{\Q[x]} \Q[x] / (x + 1) \iso  (\Q[x] / (x + 1)) / ((x - 1) \cdot (\Q[x] / (x + 1)))\] But $x - 1$ and $x + 1$ are relatively prime, since \[\left(\frac{1}{2}\right)(x + 1) + \left(\frac{-1}{2}\right)(x - 1) = \frac{x}{2} + \frac{1}{2} - \frac{x}{2} + \frac{1}{2} = 1\] Therefore $(x - 1) \cdot (\Q[x] / (x + 1)) \iso (\Q[x] / (x + 1))$ and so the tensor product is zero.

\subsubsection*{$\star$2. Consider a commutative ring $A$ and an unknown element $x \in A$. Which of the following systems of congruences has a solution for any choice of $a, b \in A$? (Hint: Chinese remainder theorem.)}
By the Chinese remainder theorem, solutions to \[x \equiv a \bmod m_1\\ x \equiv b \bmod m_2\] only obtain when $a = b \bmod \gcd(m_1, m_2)$. Since $a, b$ are going to be arbitrarily chosen from $A$, we therefore want the two moduli $m_1, m_2$ to be relatively prime.

In the first case, $A = \Z, m_1 = 6, m_2 = 35$. Since $\gcd(6, 35) = 1$, there exists a solution for any $a, b \in A$.
Secondly, we have $A = \Z, m_1 = 6, m_2 = 9$. Since $\gcd(6, 9) = 3 > 1$, there is some $a, b$ for which no solution exists.

Now we consider polynomial rings over finite fields. In the third case, $A = \F_p[x], m_1 = x^p - 1, m_2 = x - 1$. But $\gcd(x^p - 1, x - 1) = x - 1 \neq 1$ since $x^p - 1 = (x - 1)(x^{p-1} + x^{p-2} + \dotsb + x + 1)$, and hence there is some $a, b$ for which no solution exists.
Finally, we have $A = \F_2[x], m_1 = x^2 - 1, m_2 = x^2 + x + 1$. 

\subsubsection*{3. Which of the following algebras are products of fields (maybe with only one factor)?}
We use the theorem on the structure of finite $K$-algebras to decompose these algebras into products of quotients by (powers of) maximal ideals. To be products of \emph{fields}, each power must be exactly $1$.

In the first case, we have $\Z / 6\Z$. The maximal ideals here are $(2)$, $(3)$, so by the structure theorem: $\Z / 6\Z \iso \Z / 2\Z \times \Z / 3\Z$, since $(2)(3) = (6)$.

\subsubsection*{4. Which of the following statements are true?}
A finite algebra $A$ over a field $k$ has finitely many maximal ideals.
A finite algebra $A$ over a field $k$ is always an integral domain.
A finite algebra $A$ over a field $k$ is isomorphic to a product of fields.
A finite algebra $A$ over a field $k$ is isomorphic to a product $A/\mf{m}_1^{n_1} \times \dotsb \times A / \mf{m}_r^{n_r}$ for some maximal ideals $\mf{m}_1,\dotsc, \mf{m}_r$ and some integers $n_1, \dotsc, n_r$.

\subsubsection*{5. Let $k$ be a field, $A$ a $k$-algebra of dimension $2$. Choose $x \in A$, $x \notin k \sbs A$. Then $A$ is generated by $1$ and $x$ and so there is an isomorphism of algebras $k[x]/(x^2 + ax + b) \iso A$ for some $a, b \in k$. Which of the following statements are true?}

Let $L = k[x] / (x^2 + ax + b)$. Since $[A : k] = 2$, we need $[L : k] = 2$. 

Now consider the ``best case'' for $A$: suppose $k$ is algebraically closed (e.g., $k = \C$). Then there are two possibilities for $A$. We automatically need multiplication on $A$ to include $1 \cdot 1 = 1, 1 \cdot x = x, x \cdot 1 = x$, so we just have a choice for $a \cdot a$: either $a \cdot a = 1$ or $a \cdot a = 0$.

\paragraph*{Case 5a: $k = \Q$.}
\begin{proof}[Claim: There are \emph{infinitely many} non-isomorphic possibilities for $A$.]

\end{proof}

\paragraph*{Case 5b: $k = \F_p$.}
\begin{proof}[Claim: There are \emph{xxx} possibilities for $A$, up to isomorphism.]

\end{proof}

\paragraph*{Case 5c: $k$ is algebraically closed.}
\begin{proof}[Claim: There are \emph{two} possibilities for $A$, up to isomorphism.]
False. If $k$ is algebraically closed, then both $\alpha$ and $\beta$ are in $k$. We can then send $\alpha \mapsto x$ or $\beta \mapsto x$.
\end{proof}
