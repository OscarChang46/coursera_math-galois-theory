\section{Lecture Notes: 21 Mar -- 28 Mar}

\subsection{Structure of finite $K$-algebras, examples (cont'd)}
Last time we have seen that a finite algebra over a field was a product of certain quotients by powers of maximal ideals. 
Such a $k$-algebra $A$ has only finitely many maximal ideals $\mf{m}_1, \dotsc, \mf{m}_r$ and is isomorphic to $A / \mf{m}_1^{k_1} \times \dotsb A / \mf{m}_r^{k_r}$; this is a sort of ``generalized form" of the Chinese Remainder Theorem.
For example, if $A = K[x] / (F)$ where $F$ is a (not necessarily reducible) polynomial, we can decompose $F = P_1^{k_1} \dotsb P_r^{k_r}$; then by the Chinese Remainder theorem we have \[ A \iso K[x] / (P_1^{k_1}) \times \dotsb \times K[x] / (P_r^{k_r})\] 
Here, every such factor is $A / \mf{m_i}^{k_i}$, where $\mf{m_i}$ is the ideal generated by $P_i$ (mod $F$).
So now let us give a couple of definition.
\begin{dfn} 
An algebra $A$ is called \term{reduced} if it has no nilpotents (recall that $x \in A$ is nilpotent if $x \neq 0$ but $x^k = 0$ for some $k$).
This is the same as saying that, in the decomposition $A / \mf{m}_1^{k_1} \times \dotsb A / \mf{m}_r^{k_r}$, all the $k_i$ are equal to $1$.
This is the same as saying that $A$ is a product of fields $A / \mf{m}_i$.
\end{dfn}

\begin{dfn}
An algebra $A$ is called \term{local} if it has only one maximal ideal, so $A \iso A / \mf{m}^{k}$.
(Here there are lots of nilpotents; all elements of $\mf{m}$ are nilpotents, so each $x \in A$ is a unit, zero, or nilpotent.)
\end{dfn}
These definitions extend to non-finite extensions, but we lose the structure theorems; hence (e.g.) we cannot say that every element in a non-finite local extension is a unit, zero, or nilpotent.

Last time we saw that $\C \tensor{\R} \C$, $\Q(\sqrt{2} \tensor{\Q} \Q(\imath)$, etc. were reduced: $\C \tensor{\R} \C = \C \times \C$, $\Q(\sqrt{2} \tensor{\Q} \Q(\imath)$ is a field, and so on.
If we start producing similar examples, mostly they are reduced.
Why?
The presence of nilpotents reflects inseparability.

Let $K$ be a field of characteristic $p$, and consider the field of rational functions $K(X)$ as an extension of $K(X^p)$, which we denote $K(Y)$ for simplicity of notation.
Now find the tensor product $K(X) \tensor{K(Y)} K(X)$.
This is the same thing as $K(X) \tensor{K(Y)} K[T] / (T^p - Y) \iso K(X)[T] / T^p - X^p \iso K(X)[T] / (T - X)^p$.
This ring has a lot of nilpotents!
For instance, $T - X$, since $K(X)$ is a purely inseparable extension of $K(Y)$.

\subsection{Separability and base change.}
So, what is the reason for such a mysterious connection between the presence of nilpotents and separability?
Recall that separable extension $L$ over a field $K$ has a maximal possible number of homomorphisms into the algebraic closure; in fact, equal to the degree of $L$ over $K$.
This is clear, because if we have a polynomial with distinct roots then its stem field (for instance) has exactly this number of homomorphisms into the homomorphisms.
If some roots coincide, then the number of homomorphisms diminishes.
Now recall the base-change formula:
If we have $L$, $E$ extensions of $K$, with $L$ finite over $K$, then $\Hom{K}{L, E} \iso \Hom{E}{L \tensor{K} E, E}$.
Now $L \tensor{K} E = A$ is a finite $E$-algebra, and so $A \iso A / \mf{m}_1^{k_1} \times \dotsb \times A/\mf{m}_r^{k_r}$.
Now define $A_{\mr{red}}$ (``$A$-reduced") by $A_{\mr{red}} := A / \mf{m}_1 \times \dotsb A / \mf{m}_r$.
We see that $A_{\mr{red}} = A / \mc{N}(A)$, where $\mc{N}(A)$ is the ideal generated by the nilpotent elements of $A$.
Then it is clear that if we look at the homomorphisms $\Hom{E}{A, E}$, this is the same as the homomorphisms $\Hom{E}{A_{\mr{red}}, E}$ since any homomorphism into a field must be zero on all nilpotents!
Therefore we see that if there are nilpotents in the tensor product, then there is somehow ``less space" for homomorphisms, giving us the following ``slogan":
``If $A$ is not reduced, then $[A_{\mr{red}} : E] < [A : E]$, so the maximal number of homomorphisms is attained when $A$ is reduced and all quotients $A / \mf{m}_i \iso E$."
In general, the quotients $A / \mf{m}_i$ are extensions of $E$.
$A \iso A / \mf{m}_1 \times \dotsb \times A / \mf{m}_r$
$\Hom{E}{A / \mf{m}_i, E} = \{ 0 \}$ if $[A / \mf{m}_i : E] > 1$, because an $E$-homomorphism of fields which are extensions of $E$ must be injective.

Now let us take $E = \overline{K}$. 
Then $A / \mf{m}_i \iso E$ automatically since an algebraically closed field has no non-trivial finite extensions.
We have $A = L \tensor{K} \overline{K}$, $A_{\mr{red}} = \prod_{r \text{times}} \overline{K}$ and $A = A_{\mr{red}}$ if and only if $r$ is maximal, equal to $[L : K] = [A : \overline{K}]$. 
This $r$ is also equal to the number of homomorphisms $\Hom{\overline{K}}{A, \overline{K}}$, which is also equal to the number of homomorphisms $\Hom{K}{L, \overline{K}}$.
We can now formulate this result as a theorem.

\begin{thm}
Let $L$ be a finite extension of $K$.
(1) $L$ is separable if and only if $L \tensor{K} \overline{K}$ is reduced, and purely inseparable if and only if $L \tensor{K} \overline{K}$ is local;
(2) $L$ is separable if and only if for all algebraic extensions $\Omega$, we have $L \tensor{K} \Omega$ reduced, and purely inseparable if and only if $L \tensor{K} \Omega$ is local;
(3) If $L$ is separable, then $\phi: L\tensor{K} \overline{K} \to \overline{K}^n$ such that $\phi(\l \tensor k) = ( k \phi_1(l), \dotsc, k \phi_n(l) )$, where $\phi_i$ are distinct homomorphisms $L \to \overline{K}$, is an isomorphism.
\end{thm}
\begin{proof}
(1) We have seen that if $L$ is separable, this is the same thing as saying that $A = L \tensor{K} \overline{K}$ has $[L:K]$ factors $\overline{K}$. 
This is equivalent to saying that $A$ is reduced, since the dimension of $A$ over $\overline{K}$ is also equal to $[L : K]$.
If $L$ is purely inseparable, then there is only one homomorphism of $L$ into $\overline{K}$, so $A$ has only one homomorphism into $\overline{K}$; but this means that there is only one factor, which is to say that $A$ is local.

(2) If $\Omega$ is an algebraic extension, then $L \tensor{K} \Omega$ embeds into $L \tensor{K} \overline{\Omega} = L \tensor{K} \overline{K}K$ as a subring. 
One can easily check that a subring of a reduced algebra is reduced, and similarly a subring of a local algebra is local.

(3) Exercise.
\end{proof}

\begin{rmk}
In general, for modules $M$, $N$, $P$ over a ring $R$, it is \emph{not} true that if $M \embed N$ then $M \tensor{R} P \embed N \tensor{R} P$. 
If $R$ is a field, i.e. all our modules are vector spaces, then this becomes true.
\end{rmk}

\subsection{Primitive element theorem.}

\begin{thm}
Let $L$ be a finite separable extension of $K$.
Then it has only finitely many sub-extensions $K \sbs E \sbs L$.
\end{thm}
\begin{proof}
Let $E$ be sub-extension. 
Perform a base-change to $E \tensor{K} \overline{K} \embed L \tensor{K} \overline{K}$; this is a (reduced) $\overline{K}$-subalgebra.
Moreover, $E \tensor{K} \overline{K} \iso \overline{K}^m$ and $L \tensor{K} \overline{K} \iso \overline{K}^n$.
We know that $\overline{K}$ is generated by \term{idempotents} ($x$ such that $x^2 = x$), namely, these are just $(0, 0, \dotsc, 1, 0, \dotsc, 0)$ with the $1$ in the $i$th place for $i = 1, \dotsc, m$.
On the other hand, $L \tensor{K} \overline{K} \iso \overline{K}$ has only finitely many idempotents: $(a_1, \dotsc, a_i, \dotsc, a_n)$ is idempotent if and only if all $a_i$ are either $0$ or $1$.
Hence, there are only finitely many ways of generating subalgebras this way.
\end{proof}

Now we state the ``Primitive element theorem" as a corollary.

\begin{cor}
Let $L$ be a finite separable extension.
Then there exists $\alpha \in L$ such that $L = K(\alpha)$.
\end{cor}
\begin{proof}
If $L$, $K$ are infinite, then $L$ cannot be a finite union of proper sub-extensions (a vector space over an infinite field is not a finite union of subspaces).
If $L$, $K$ are finite, we have described all finite extensions, and have seen that they are generated by one element.
\end{proof}

We now look at two examples.

\begin{ex}
(1) Take $\Q(\sqrt{2}, \sqrt{3}) = \Q(\sqrt{2} + \sqrt{3})$. Since $[\Q(\sqrt{2}, \sqrt{3}) : \Q ] = 4$, all subextensions are quadratic, and no quadratic polynomial has $\sqrt{2} + \sqrt{3}$ for a root, it must be a primitive element.

(2) (Counter-example) Let $K = \F_p$ and consider $K(X, Y)$ as an extension of $K(X^p, Y^p)$.
This has degree $p^2$.
Now any $\alpha \in K(X, Y) \ K(X^p, Y^p)$ is purely inseparable of degree $p$ over $K(X^p, Y^p)$, since $\alpha^p \in K(X^p, Y^p)$, so no element like this can generate our extension.
\end{ex}

\subsection{Normal extensions.}

\begin{dfn}
A \term{normal extension} of $K$ is a splitting field of a family of polynomials in $K[x]$.
\end{dfn}

For instance, a splitting field of a single polynomial is normal.

\begin{thm}
The following conditions are equivalent for an extension $L$ of $K$:
(1) For any $x \in L$, the minimal polynomial $P_\mr{min}(x, K)$ splits in $L$;

(2) $L$ is normal;

(3) All homomorphisms from $L$ to $\overline{K}$ have the same image;

(4) $\Aut{L/K}$ acts transitively on $\Hom{K}{L, \overline{K}}$.
\end{thm}
\begin{proof}
(1) $\implies$ (2):  Take $(P_i)_{i \in I} = \set{P_\mr{min}(x, K) | x \in L}$. 
Then $L$ is a splitting field of this family.

(2) $\implies$ (3): Let $S$ be the set of roots of $P_i, i \in I$ in $L$, and $S^\prime$ the set of roots of $P_i, i \in I$ in $\overline{K}$.
Any $\phi: L \to \overline{K}$ sends $S$ to $S^\prime$, but $S$ generates $L$ over $K$, so $\phi(S)$ determines $\phi(L)$.

(3) $\implies$ (4): If $j, j^\prime \in \Hom{K}{L, \overline{K}}$ then they are isomorphisms from $L$ to its image $L^\prime$. 
Hence we can produce $j^{-1} \after j^\prime: L (\to L^\prime) \to L$, that is, $j^{-1} \after j^\prime \in \Aut{L/K}$ and it sends $j$ to $j^\prime$.

(4) $\implies$ (1): Consider $P_\mr{min}(x, K)$ with roots $\alpha_1, \dotsc, \alpha_n \in \overline{K}$.
A map $K(x) \to K(\alpha_i)$ extends to $j_i: L \to \overline{K}, x \mapsto \alpha_i$ by the theorem on extensions of homomorphisms.
Now there exist $\theta_i \in \Aut(L / K)$ such that $j_1 \theta_i = j_i \to \alpha_i \in j_1(L)$, hence all roots are in $j_1(L)$ and the polynomial splits over $j_1(L)$.
But this means that the polynomial also splits over $L$.
\end{proof}

\subsection{Galois extensions.}
We are now ready to give the main definition of this course.

\begin{dfn}
A \term{Galois extension} is a normal and separable extension.
\end{dfn}

This will be the central object of Galois theory.

\begin{thm}
Let $L$ be a finite extension of $K$. Then the number of automorphisms of $L$ over $K$ is less than or equal to $[L : K]$, with equality if and only if $L$ is Galois.
\end{thm}
\begin{proof}
We know that $\Aut{L/K}$ acts freely on $\Hom{K}{L, \overline{K}}$.
So the number of automorphisms is equal to the cardinality of an orbit of this action, which is less than or equal to the cardinality of the set itself.
We have equality if and only if the action is transitive, and we have just seen in the previous theorem that this means that $L$ is normal over $K$.
Then the size of $\Aut{L/K}$ is less than or equal to the size of $\Hom{K}{L, \overline{K}}$ (equality iff normal), which is less than or equal to $[L : K]$ (equality iff separable).
Therefore the size of $\Aut{L/K}$ is less than or equal to $[L : K]$, with equality iff Galois.
\end{proof}

\begin{rmk} 
Some remarks on normal extensions. 
Let $L / K$ be normal.

(1) Let $\phi: L_1 \isomap L_2$ be an isomorphism of subextensions.
Then $\phi$ extends to an automorphism of $L$.
To see this, we embed $L \embed \overline{K}$ and remark that $\phi$ extends to a map into $\overline{K}$ but all such extended maps have the same image, namely $L$.

(2) The group $\Aut{L/K}$ acts transitively on the roots of any irreducible polynomial in $K[x]$.
Again, an isomorphism of stem fields extends to an isomorphism of $L$.

(3) If $\Aut{L/K}$ fixes $x \not\in K$ then $x$ is purely inseparable over $K$.
This is clear because if so, $P_\mr{min}(x, K)$ must have $x$ as the only root.
In particular, if $L$ is Galois, then $L^{\Aut{L/K}} = K$.
(Notation: If $G$ is a group acting on a set $X$, then $X^G = \set{x \in X : gx = x \forall g \in G}$ is the set of invariants.
\end{rmk}

\begin{dfn}
If $L$ is Galois, the \term{Galois group} $G = \Gal{L/K}$ is just $\Aut{L/K}$.
(Then $L^{\Gal(L/K)} = K$.)
\end{dfn}

\subsection{Artin's theorem.}

So, motivated by this remark---that the field of invariants of $L$ under the action of $G$ is $K$---we formulate and prove an important theorem.

\begin{thm}[Artin]
Let $L$ be a field, and $G \sbs \Aut{L}$.

(1) If $G$ acts with finite orbits (i.e., all orbits of $G$ are finite), then $L$ is a Galois extension of $L^G$;

(2) If $|G| = n$ then $[L : L^G] = n$, and $G$ is the Galois group.
\end{thm}
\begin{rmk}
Notice that finite orbits and finiteness are \emph{not the same thing}! 
It's typical for Galois groups to act with finite orbits: if $G = \Gal{L/K}$ and $x \in L$ is a root of a polynomial of some finite degree, its splitting field is finite over $K$, so the orbit of $x$ is also finite.
(It consists of roots of $P_\mr{min}(x, K)$.)
But $\Gal{L/K}$ can be infinite when $L$ is not finite over $K$: for instance, if $K = \F_p$ and $L = \overline{\F}_p$.
\end{rmk}
\begin{proof}[Proof of Artin's theorem.]
(1) Take $x = x_1 \in L \ L^G$ and let $x_1, x_2, \dotsc, x_k$ be the orbit of $x$.
Now $P(x) = \prod_{i = 1}^{k} (x - x_i)$ is $G$-invariant!
Then $P \in L^G[x]$, $P$ is separable (all $x_i$ are distinct), and $L$ is a splitting field of $P$.
Therefore $L$ is Galois over $L^G$.

(2) Suppose that $|G| = n$. Then the size of any orbit is less than or equal to $n$.
Take $x$ as above: then $[L^G(x) : L^G] \leq n$.
We claim that this implies $[L : L^G] \leq n$.
If we already knew that $L$ was finite over $L^G$, this would be very easy: in fact, a direct consequence of the primitive element theorem.
We don't know yet, though, that $L$ is finite.
Indeed, take $x$ such that $[L^G(x): L^G]$ is maximal, and take $y \in L$.
Then $L^G(x, y)$ is finite over $L$, and we can apply the primitive element theorem: $L^G(x, y) = L^G(z)$.
But $[L^G(x): L^G] \geq [L^G(z): L^G]$, hence $L^G(x) = L^G(z)$, so $y \in L^G(x)$.
Since we can do this for any $y$, we can conclude that $L = L^G(x)$.
In particular, $[L : L^G] \leq n$.
Now, if $[L : L^G] < n$, then $L$ cannot have $n$ automorphisms over $L^G$, but $G \sbs \Aut{L/L^G}$, a contradiction.
Therefore we conclude that $[L : L^G] = n$, and $G = \Aut{L/L^G}$.
\end{proof}