\subsection[Quiz]{Week 1 Quiz}

\subsubsection*{1. Which of the following are true?}
\paragraph*{A finite extension of fields is algebraic.} 
This is \emph{true}.

\paragraph*{An algebraic extension of fields is finite.} T
his is \emph{false}; for example, the field of all algebraic numbers is an infinite extension of $\Q$.

\paragraph*{A finitely generated and algebraic extension of fields is finite.}
This is \emph{true}.

\subsubsection*{2. Which of the following pairs is an extension of fields?}
$\Z, \Q$ is \emph{not} a field extension because $\Z$ is not a field.

$\Q, \R$ is a field extension because $\R$ is a field and $\Q \sbs \R$.

$\Q(\imath), \R$ is \emph{not} a field extension because, e.g., $\imath \in \Q(\imath)$ but $\imath \notin \R$, and so $\Q(\imath)$ is not a subfield of $\R$.

$\Q(\imath), \C$ is a field extension because $\C$ is a field and $\Q(\imath) \sbs \C$.

\subsubsection*{3. What is the minimal polynomial of $e^{2\pi\imath/3}$ over $\Q$?}
Let $\zeta = e^{2\pi\imath/3}$, and note that $\zeta^3 = e^{2\pi\imath} = 1$. Therefore $\zeta$ is a root of the polynomial $Q(x) = x^3 - 1$. Now $Q$ is not irreducible: $Q = PR$, where $P(x) = x^2 + x + 1$ and $R(x) = x - 1$. $R(\zeta) \neq 0$ but $P(\zeta) = 0$, and $P$ is irreducible over $Q$ (by, e.g., the quadratic formula). Therefore $P(x) = x^2 + x + 1$ is the minimal polynomial for $\zeta$ over $\Q$.

\subsubsection*{4. Which of the following polynomials $f$ is irreducible over the specified field $K$?}
$f_1 = x^2 + x + 1$ is irreducible over $K_1 = \Q$; see previous question.

$f_2 = x^2 - 2$ is irreducible over $K_2 = \Q$, since its roots are $\pm \sqrt{2} \notin \Q$.

$f_3 = x^2 - 2$ is \emph{not} irreducible over $K_3 = \R$, since its roots are $\pm \sqrt{2} \in \R$.

$f_4 = x^2 + x + 1$ is \emph{not} irreducible over $K_4 = \F_3$: we have $f_4(1) = 1 + 1 + 1 = 0$ since the field has characteristic $3$, and $1 \in \F_3$.

$f_5 = x^4 + 6x^2 + 2$ is irreducible over $K_5 = \Q$. Setting $y = x^2$ and $\hat{f}_5 = y^2 + 6y + 2$, we obtain by the quadratic formula \[y = \frac{-6 \pm \sqrt{36 - 4}}{2} = \frac{-6 \pm \sqrt{32}}{2} = \frac{-6 \pm 4\sqrt{2}}{2} = -3 \pm 2 \sqrt{2},\] and hence $x = \pm \sqrt{-3 \pm 2 \sqrt{2}} \notin \Q$.

$f_6 = x^3 - 1$ is \emph{not} irreducible over $K_6 = \Q$; see previous question.

\subsubsection*{5. Which of the following quotient rings is a field?}
Note that this is equivalent to asking if the polynomial we're modding out by is irreducible over the base field.

$\R[x]/(x^2 - 2)$ is \emph{not} a field, since $x^2 - 2$ is not irreducible over $\R$.

$\Q[x]/(x^2 - 2)$ is a field, since $x^2 - 2$ is irreducible over $\Q$.

$\F_3[x]/(x^2 + x + 1)$ is \emph{not} a field, since $x^2 + x + 1$ is not irreducible over $F_3$.

$\R[x]/(x^2 - 1)$ is \emph{not} a field, since $x^2 - 1$ is not irreducible over $\R$.

$\R[x]/(x^2 + 1)$ is a field, since $x^2 + 1$ is irreducible over $\R$.

\subsubsection*{6. What is the degree of the field extension $\Q \sbs \Q(\sqrt{2}, \sqrt{3})$?}
We know that the extension is generated by products of $1, \sqrt{2}, \sqrt{3}$. Now $1^2 = 1$, $(\sqrt{3})^2 = 3$, $(\sqrt{2})^2 = 2$, and $\sqrt{2}\sqrt{3} = \sqrt{6}$; therefore any element $q \in \Q(\sqrt{2}, \sqrt{3})$ can be written $q = a + b\sqrt{2} + c\sqrt{3} + d\sqrt{6}$ with $a, b, c, d \in \Q$. Therefore $[\Q(\sqrt{2}, \sqrt{3}) : \Q] = 4$.