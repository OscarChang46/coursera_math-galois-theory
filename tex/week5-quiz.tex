\subsection[Quiz]{Week 5 Quiz}

\subsubsection*{$\star$ 1. Fix a field $k$ and a finite $k$-algebra $A$. Which of the following are true?}

\paragraph*{A finite $k$-algebra $A$ is reduced (has no nilpotent elements) if and only if it is a field.} This is \emph{false}; for example, %TODO

\paragraph*{Let $A \iso A / \mf{m}_1^{n_1} \times \dotsb \times A / \mf{m}_r^{n_r}$, where $\mf{m}_1,\dotsc,\mf{m}_r$ are the maximal ideals in $A$. Let $A_{\mathrm{red}}$ be the reduction of $A$, that is, the quotient of $A$ by te ideal of nilpotent elements. Then $A_{\mathrm{red}} \iso A / \mf{m}_1 \times \dotsb \times A / \mf{m}_r$.} This is \emph{true}; by %TODO

\paragraph*{Let $L / k$ be a finite field extension of degree $d$. Then the number of $k$-homomorphisms $L \to \bar{k}$ is $d$.} This is \emph{false} in general, only true when $L$ is separable; %TODO

\paragraph*{Let $L / k$ be a finite field extension of degree $d$. Then the number of $k$-homomorphisms $L \to \bar{k}$ is $r$, where $r$ is the number of maximal ideals in $A = L \tensor{k} \bar{k}$.} This is \emph{true}; %TODO

\paragraph*{Let $L / k$ be a finite field extension. It is separable if and only if $L \tensor{k} \bar{k}$ is a reduced $\bar{k}$-algebra.} This is \emph{true}; %TODO

\subsubsection*{$\star$ 2. Recall that if $L / k$ is a finite separable field extension, then $L = k(\alpha)$ for some $\alpha \in L$. For which fields does equality hold?}

\paragraph*{$\Q(\sqrt{2}, e^{2\pi\imath / 3}) = \Q(\sqrt{2} + e^{2\pi\imath / 3})$.} This is \emph{true}; we know that since $\Q$ has characteristic $0$ any extension is separable, so the primitive element theorem must hold in this case. More specifically, let $\gamma = \sqrt{2} + e^{2 \pi \imath / 3}$. Then:
\begin{align*}
\gamma^1 &= \sqrt{2} + e^{2 \pi\imath / 3} \\
%
\gamma^2 &= 1 + (-1 + 2 \sqrt{2}) e^{2 \pi\imath / 3}\\
%
\gamma^3 &= (1 - \sqrt{2}) + (6 - 3\sqrt{2}) e^{2 \pi\imath / 3}\\  
%
\gamma^4 &= -8 + 4\sqrt{2} + (-11 + 8\sqrt{2}) e^{2 \pi\imath / 3}\\
%
\gamma^4 - 4 \gamma^2 + 12 &= -8 + 4\sqrt{2} + (-11 + 8\sqrt{2}) e^{2 \pi\imath / 3}
                              -4 + (4 - 8 \sqrt{2}) e^{2 \pi\imath / 3} + 12\\
                           &= 4\sqrt{2} - 7 e^{2 \pi\imath / 3}
\end{align*}
And we see that from here, we can obtain $-(\gamma^4 - 4 \gamma^2 - 4 \gamma + 12)/11 = e^{2 \pi \imath / 3}$ and $(\gamma^4 - 4 \gamma^2 + 7 \gamma + 12)/16 = \sqrt{2}$. Therefore indeed $\Q(\sqrt{2}, e^{2 \pi\imath / 3}) = \Q(\sqrt{2} + e^{2 \pi\imath / 3})$.

\paragraph*{$\Q(\sqrt[3]{2}, e^{2\pi\imath / 3}) = \Q(\sqrt[3]{2} \cdot e^{2\pi\imath / 3})$.} This is \emph{false}; writing $\sqrt[3]{2} \cdot e^{2\pi\imath / 3} = 2^{1/3} \cdot (-1)^{2/3} = \eta$, we see:
\begin{align*}
\eta^1 &= 2^{1/3} \cdot (-1)^{2/3}\\
%
\eta^2 &= -2^{2/3} \cdot (-1)^{1/3}\\
%
\eta^3 &= 2
\end{align*}
Hence $[\Q(\sqrt[3]{2}\cdot e^{2\pi\imath / 3} : \Q] = 3$, and yet $\set{1, 2^{1/3}, 2^{2/3}, e^{2\pi\imath / 3}, e^{4\pi\imath / 3}}$ is a basis of $\Q(\sqrt[3]{2}, e^{2\pi\imath / 3})$, making $[\Q(\sqrt[3]{2}, e^{2\pi\imath / 3}) : \Q] = 5$.

\paragraph*{$\F_{2^3} = \F_{2}(\zeta)$, where $\zeta$ is a primitive third root of unity.} This is \emph{false}; we know $\F_{2^3} \iso \F_2[x] / (x^3 + x^2 + x + 1)$ and $x^3 + x^2 + x + 1 = (x^4 - 1) / (x - 1)$, but this doesn't have $\zeta$ as a root. Therefore $\zeta \notin \F_{2^3}$, so $\F_{2^3} \neq \F_2(\zeta)$.

\paragraph*{$\F_{2^2} = \F_{2}(\zeta)$, where $\zeta$ is a primitive third root of unity} This is \emph{true}; we know that $\F_{2^2} \iso \F_2[x] / (x^2 + x + 1)$, and $x^2 + x + 1 = (x^3 - 1) / (x - 1)$ is the third cyclotomic polynomial so it has $\zeta$ as a root. Therefore $\F_{2^2} = \F_2(\zeta)$.

\subsubsection*{$\star$ 3. Which of the following statements are true?}
\paragraph*{$\F_{p^n} / \F_p$ is a Galois extension.} This is \emph{true}; %TODO

\paragraph*{$\Q(\sqrt[3]{2}, e^{2\pi\imath / 3}) / \Q$ is a normal extension.} This is \emph{true}; %TODO

\paragraph*{$k(x) / k$, where $x$ is an indeterminate, is a Galois extension of $k$.} This is \emph{false}; %TODO

\paragraph*{$\Q(\sqrt[3]{2})/\Q$ is a normal extension.} This is \emph{false}; $\Q(\sqrt[3]{2})$ is a stem field for the polynomial $x^3 - 2$, but is not a splitting field. In particular the complex third roots of unity (say, $\zeta, \zeta^2$) are not in this field, yet $\zeta \cdot \sqrt[3]{2}$ is a root of the polynomial.

\subsubsection*{$\star$ 4. Which of the following statements are true?}
\paragraph*{Every finite extension of $\F_p$ is Galois.} This is \emph{true}; %TODO

\paragraph*{$|\Aut{\F_{p^n} / \F_p}| = n!$} This is \emph{xxx}; the polynomial $x^{p^n} - x$ has $p^n$ distinct roots, each of order $p^n$, so 

\paragraph*{If $L / k$ is a finite extension, then $|\Aut{L / k}| \leq [L : k]$, with equality if an only if $L / k$ is Galois.}

\paragraph*{$|\Aut{\Q(\sqrt[3]{2}) / \Q}| = 1$.} This is \emph{true}; the set of automorphisms must permute the roots of $x^3 - 2$, but only one such root is an element of $\Q(\sqrt[3]{2})$.

\paragraph*{$|\Aut{\Q(\sqrt[3]{2}) / \Q}| = 3$.} This is \emph{false}.

\paragraph*{$\Q(\sqrt[3]{2})/ \Q$ is Galois.} This is \emph{false}.

\subsubsection*{$\rightarrow$ 5. Which of the following statements are true?}

\paragraph*{Let $F \subset \C$ be a subfield stable under complex conjugation. Then $F / F \cap \R$ is a Galois extension.} This is \emph{false}; consider $F = \Q(\sqrt[3]{2} \cdot e^{2\pi\imath / 3})$. Then since $(e^{2\pi\imath / 3})^2 = e^{4\pi\imath / 3} = e^{-2\pi\imath / 3} = \overline{(e^{2\pi\imath / 3})}$, the field is stable under complex conjugation. Yet $F / F \cap \R = \Q(\sqrt[3]{2} \cdot e^{2\pi\imath / 3}) / \Q$ which is not a separable extension.

\paragraph*{Let $F \subset \C$ be a subfield. Then $F / F \cap \R$ has degree $2$.} This is \emph{false}; consider $F = \Q(\sqrt[3]{2}, e^{2\pi\imath / 3}) = \Q[x] / (x^3 - 2)$. Then $F \cap \R = \Q(\sqrt[3]{2})$ and hence $F / F \cap \R = \Q(\sqrt[3]{2}, e^{2\pi\imath / 3}) / \Q(\sqrt[3]{2}) = \Q(\sqrt[3]{2})(e^{2\pi\imath / 3}$ which is a degree $3$ extension.

\paragraph*{An algebraic closure $\bar{Q}$ is Galois over $\Q$.} This is \emph{true}; %TODO

\paragraph*{An algebraic closure $\bar{\F_p}$ is Galois over $\Q$.} This is \emph{true}; %TODO

\paragraph*{An algebraic closure $\overline{\F_p(T)}$ is Galois over $\F_p(T)$.} This is \emph{false}; %TODO

\paragraph*{$\Gal{\bar{\Q} / \Q}$ acting on $\bar{\Q}$ has an infinite orbit.} This is \emph{true}; %TODO

