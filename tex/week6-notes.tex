\section[Galois correspondence and first examples.]{Week 6 Lecture Notes: }
Last time we defined $L/K$ is Galois if and only if separable and normal, which is equivalent to saying that $L$ is a splitting field of a family of separable irreducible polynomials over $K$.
This was nontrivial and required a proof, but this is equivalent to saying that (for finite extensions) that the number of automorphisms of $L$ over $K$ is equal to the degree of $L$ over $K$.

\subsection{Some further remarks on normal extensions.}
We have seen, for instance, that an extension $L \sps M \sps K$ is finite or algebraic or separable or purely inseparable if and only if it was true for $L \sps M$ and $M \sps K$.
This is \emph{no longer true} for normality: if we have a tower of extensions $K \sbs L \sbs M$, then $M$ normal over $K$ implies $M$ normal over $L$, since if $M$ is a splitting field of polynomials over $K$, one can just consider those polynomials as elements of $L[x]$.
But $L$ does not have to be normal over $K$, because $L$ can be just a stem field of a polynomial of which $M$ is a splitting field, and when they are not equal $L$ is not normal by definition.
For example, with $\Q \embed \Q(\sqrt[4]{2}) \embed \Q(\sqrt[4]{2}, \imath)$, the final extension is the splitting field of $x^4 - 2$, while the intermediate extension is just a stem field of $x^4 - 2$, and not normal over $\Q$.
The polynomial $x^4 - 2$ has two roots in this extension but also has two roots outside the extension.

(2) A quadratic extension is normal.
In fact, a quadratic extension is the stem field of a quadratic polynomial, but we know how to solve quadratic equations, and so we know that both roots must be in the same field.

(3) One often has $K \sbs L \sbs M$ where $L$ is normal over $K$ and $M$ is normal over $L$ but $M$ is not normal over $K$.
If you take quadratic extensions, then in most cases their composition will not be normal over the base field.
For example, $\Q \sbs \Q(\sqrt{2}) \sbs \Q(\sqrt[4]{2})$ is such that $\Q(\sqrt{2})$ is normal over $\Q$ and $\Q(\sqrt[4]{2})$ is normal over $\Q(\sqrt{2})$, but we have already seen that $\Q(\sqrt[4]{2})$ is not normal over $\Q$.

We have also seen that if $L$ is a field and $G \sbs \Aut{L}$ is a group of automorphisms, then we can consider $L^G$ the fixed field.
If we have $K \sbs L$ a subfield, we can consider the group of automorphisms $\Aut{L/K}$ in the case when $L$ is normal.
If $L$ is separable over $K$, then the fixed field $L^{\Aut{L/K}} = K$ because the group of automorphisms was permuting the roots of the minimal polynomial, etc.
If $G$ is finite, then $L$ is Galois over $L^G$ and the degree $[L:L^G] = |G|$.
Now we summarize all these facts in a theorem, which is in fact the main fact of this lecture course---the Galois correspondence.

\subsection{The Galois correspondence.}
Let $L / K$ be a Galois extension.
The group of automorphisms $\Aut{L/K}$ is called, by definition, the Galois group $\Gal{L/K}$.

\begin{thm}[Galois]
(1) If $L$ is finite over $K$, then there is a bijection between the subextensions of $L$ and the subgroups of $\Gal{L/K}$.
The correspondence is such that we send a subextension $F$ to the subgroup $\Gal{L/F}$, and a subgroup $H$ to its fixed field $L^H$.
Now $F$ is not necessarily Galois over $K$, but one can say when it is Galois.

(2) $F$ is Galois over $K$ if and only if, for any $g \in \Gal{L/K}$, $g(F) = F$, which if and only if $\Gal{L/F}$ is a normal subgroup in $\Gal{L/K}$.
In this case, $\Gal{L/F} \to \Gal{F/K}: g \mapsto g|_F$ is surjective, and the kernel is just $\Gal{L/F}$.
\end{thm}
\begin{proof}
(1) This has been more or less done: $L^{\Gal{L/F}} = F$ is true for separable extensions.
Then also $H \sbs \Gal{L/L^H}$ by definition, but also by Artin's theorem, $[L : L^H] = |H|$, and we have seen that for normal and separable extensions, $[L:L^H] = |\Gal{L/L^H}|$.
Hence $H = \Gal{L/L^H}$.
This means that the map sending $F \to \Gal{L / F}$ and the map sending $H \to L^H$ are mutually inverse.
So, in particular, these are bijections.

(2) We have three equivalences: (i) $F$ is Galois; (ii) $\forall g, g(F) = F$; (iii) $\Gal{L / F}$ is a normal subgroup of $\Gal{L/K}$.

(i) $\implies$ (ii): Let $x \in F$. Then $P_{\mr{min}}(x, K)$ splits in $L$ and has a root in $F$.
But $F$ is normal, and hence $P_{\mr{min}}(x, K)$ splits in $F$.
Then any $g$ in the Galois group preserves $F$, since it permutes the roots of $P_{\mr{min}}(x, K)$.
Since this is true for any $x \in F$, this means that $g(F) \sbs F$, because $F$ is generated by such roots of minimal polynomials.

(ii) $\implies$ (i): If $g(F) \sbs F$, then all roots of $P_{\mr{min}}(x, K)$ for $x \in F$ are in $F$, since the Galois group acts transitively on the roots of an irreducible polynomial.
Therefore $F$ is normal (this is the definition of normality).



\end{proof}

\subsection{First examples (polynomials of degree 2 and 3).}

\subsection{Discriminant. Degree 3. Finite fields.}

\subsection{An infinite degree example. Roots of unity: Cyclotomic polynomials.}

\subsection{Irreducibility of cyclotomic polynomials. The Galois group.}
