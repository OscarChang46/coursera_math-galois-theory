\subsection[Quiz]{Week 2 Quiz}

\subsubsection*{$\star$1. Which of the following are true?}
\paragraph*{a. There is a field $K$ and a polynomial $P \in k[x]$ such that $P$ can never be written as a product of linear factors in $F[x]$, no matter how large the extension field $F/K$.}
This is \emph{false}; we proved (Thm. 4) that a splitting field for $P$ always exists.

\paragraph*{b. If $P$ is an irreducible polynomial of degree $d$ over a field $K$, then there is a splitting field of $P$ of degree at most $d$.}
This is \emph{false}; consider $P = x^3 - 2 \in \Q[x]$, so the splitting field for $P$ is $\Q(\sqrt[3]{2}, j)$ (where $j$ is the primitive third root of unity). Now the splitting field has degree 6, but the polynomial only has degree 3. 

\paragraph*{c. If $P$ is an irreducible polynomial of degree $d$ over a field $K$, then there is a splitting field of $P$ of degree at most $d!$ over $K$.}
This is \emph{true}; we proved this (Thm. 4).

\paragraph*{d. Given an irreducible polynomial $P$ of degree $d$ over a field $K$, it is possible to have a field extension $F/K$ of degree less than $d$ that contains a root of $P$.}
This is \emph{false}: Let $\alpha$ be a root of $P$. Then the stem field $K(\alpha)$ is generated by $\set{1, \alpha, \alpha^2, \dotsc, \alpha^{d-1}}$, and so has dimension (and degree over $K$) equal to $d$; if any of these elements were not in the set, then it wouldn't be a field (e.g., suppose $\alpha^2 \notin K(\alpha)$ with $d = 4$, and therefore $\alpha^3 / \alpha = \alpha^2 \notin K(\alpha)$, so the field axioms aren't satisfied).

\paragraph*{e. Let $P$ be an irreducible polynomial over a field $K$, and $K(\alpha)$, $K(\beta)$ be two stem fields for $P$ over $K$. Then it is possible for there to be two distinct isomorphisms $f,g: K(\alpha) \to K(\beta)$ with $f(\alpha) = g(\alpha) = \beta$.}
This is \emph{false}; we proved (Prop. 7) that there is a unique such isomorphism.

\subsubsection*{$\star$2. Let $F$ be a splitting field of $x^4-2$, constructed as the subfield of $\C$ generated by the roots of $x^4 - 2$. Which of the following are true?}
Note that the roots of $x^4 - 2$ are $\sqrt[4]{2}, \imath \sqrt[4]{2}, -\sqrt[4]{2}$, and $-\imath \sqrt[4]{2}$, since $\imath$ and $-\imath$ are the primitive fourth roots of unity. Therefore $F = \Q(\imath, \sqrt[4]{2})$.

\paragraph*{a. There is a subfield $E \sbs F$ of degree 4 over $\Q$ and containing a root of $x^4 - 2$.}
This claim is \emph{true}: Since $\sqrt[4]{2}$ is a root of $x^4 - 2$, let $E = \Q(\sqrt[4]{2})$. 
Then $[E: \Q] = 4$, since $E$ is generated by $\set{1, \sqrt[4]{2}, \sqrt{2}, (\sqrt[4]{2})^3}$ over $\Q$.

\paragraph*{b-d. What is the degree of $F$ over $\Q$?}
The degree of $\Q(\imath, \sqrt[4]{2})$ over $\Q$ is 8, since it's generated by $\set{1, \sqrt[4]{2}, \sqrt{2}, (\sqrt[4]{2})^3, \imath, \imath\sqrt[4]{2}, \imath\sqrt{2}, \imath(\sqrt[4]{2})^3}$, which is an eight-element basis.

\paragraph*{e. $F$ contains the complex number $\imath$.}
This claim is \emph{true}: $\sqrt[4]{2}, \imath \sqrt[4]{2} \in F$, so $\imath \sqrt[4]{2} / \sqrt[4]{2} = \imath \in F$.

\paragraph*{f-g. What is the degree of $F$ over $\Q(\imath)$?}
We know that $[\Q(\imath):\Q] = 2$ and $[F:\Q] = 8$, so we must have $[F:\Q(i)] = 4$.

\subsubsection*{3. Let $F$ be a splitting field of $x^4-2$, constructed as the subfield of $\C$ generated by the roots of $x^4-2$, and consider the group $G=\mr{Aut}(F)$ of automorphisms of the field $F$. Which of the following are true?}
Let $\alpha = \sqrt[4]{2}$. Then the roots of $x^4 - 2$ are $\alpha, \imath\alpha, -\alpha, -\imath\alpha$.

\paragraph*{$F$ is generated over $\Q$ by the real root $\alpha$ of $x^4 - 2$ and by the complex number $\imath$: $F = \Q(\alpha,\imath)$.}
This claim is \emph{true}: see question 2.

\paragraph*{There is an automorphism $\phi \in G$ with $\phi(\alpha) = \alpha, \phi(\imath) = -\imath$ and another automorphism $\psi$ with $\psi(\alpha) = \imath\alpha,\psi(\imath)=\imath$.}
The minimal polynomial for $\imath$ over $\Q$ is $P_{\min}(\imath, \Q) = x^2 + 1$, and in fact this is also the minimal polynomial for $\imath$ over $\Q(\alpha)$ since $\imath$ is imaginary. Therefore, since $F$ is a stem field for $x^2 + 1$ over $\Q(\alpha)$ that contains two roots of $x^2 + 1$ (that is, $\imath$ and $-\imath$, there must exist a $\Q(\alpha)$-automorphism taking $\imath \mapsto -\imath$, by Prop. 7.

Similarly, the minimal polynomial for $\alpha$ over $\Q$ is $P_{\min}(\alpha, \Q) = x^4 - 2$, and this is also the minimal polynomial for $\alpha$ over $\Q(\imath)$ since $\alpha$ is irrational. Therefore, since $F$ is a stem field for $x^4 - 2$ over $\Q(\imath)$ that contains two roots of $x^4 - 2$ (that is, $\alpha$ and $\imath\alpha$, there must exist a $\Q(\imath)$-automorphism taking $\alpha \mapsto \imath\alpha$, by Prop. 7.

This claim is \emph{true}: There must be a $\Q(\alpha)$-automorphism that interchanges roots (i.e. sends $\imath \mapsto -\imath$), and similarly there must be a $\Q(\imath)$-automorphism that interchanges roots (i.e. sends $\alpha \mapsto \imath\alpha$). These automorphisms are precisely $\phi$ and $\psi$.

\paragraph*{The subgroup of G generated by $\phi$ and $\psi$ is commutative.}
This claim is \emph{false}. We see that $\phi(\psi(\alpha)) = \phi(\imath\alpha) = -\imath\alpha$ whereas $\psi(\phi(\alpha)) = \psi(\alpha) = \imath\alpha$. 


Write $z \in F$ as a linear combination of its generators: \[z = q_1 + q_2 \alpha + q_3 \alpha^2 + q_4 \alpha^3 + q_5 \imath + q_6 \imath\alpha + q_7 \imath\alpha^2 + q_8 \imath\alpha^3\] Then the automorphisms $\phi$ and $\psi$ act on $F$ by: \[\phi(z) = q_1 + q_2 \alpha + q_3 \alpha^2 + q_4 \alpha^3 - q_5 \imath - q_6 \imath\alpha - q_7 \imath\alpha^2 - q_8 \imath\alpha^3\] and
\begin{align*}
	\psi(z) &= q_1 + q_2 \imath\alpha + q_3 \imath^2\alpha^2 + q_4 \imath^3\alpha^3 + 
						 q_5 \imath + q_6 \imath(\imath\alpha) + q_7 \imath(\imath^2\alpha^2) + q_8 \imath(\imath^3\alpha^3) \\
					&= q_1 + q_2 \imath\alpha - q_3 \alpha^2 - q_4 \imath\alpha^3 + q_5 \imath - q_6 \alpha - q_7 \imath\alpha^2 + q_8 \alpha^3 \\
					&= q_1 - q_6 \alpha - q_3 \alpha^2 + q_8 \alpha^3 + q_5 \imath + q_2 \imath\alpha - q_7 \imath\alpha^2 - q_4 \imath\alpha^3
\end{align*}
In fact, we can write $\phi$ and $\psi$ as matrices under the given basis:
\[
	M_\phi = \begin{bmatrix}  1 &  0 &  0 &  0 &  0 &  0 &  0 &  0 \\ 
														0 &  1 &  0 &  0 &  0 &  0 &  0 &  0 \\ 
														0 &  0 &  1 &  0 &  0 &  0 &  0 &  0 \\ 
														0 &  0 &  0 &  1 &  0 &  0 &  0 &  0 \\ 
														0 &  0 &  0 &  0 & -1 &  0 &  0 &  0 \\ 
														0 &  0 &  0 &  0 &  0 & -1 &  0 &  0 \\ 
														0 &  0 &  0 &  0 &  0 &  0 & -1 &  0 \\ 
														0 &  0 &  0 &  0 &  0 &  0 &  0 & -1 \end{bmatrix},
\qquad %
	M_\psi = \begin{bmatrix}  1 &  0 &  0 &  0 &  0 &  0 &  0 &  0 \\
														0 &  0 &  0 &  0 &  0 &  1 &  0 &  0 \\ 
														0 &  0 & -1 &  0 &  0 &  0 &  0 &  0 \\ 
														0 &  0 &  0 &  0 &  0 &  0 &  0 & -1 \\ 
														0 &  0 &  0 &  0 &  1 &  0 &  0 &  0 \\ 
														0 & -1 &  0 &  0 &  0 &  0 &  0 &  0 \\ 
														0 &  0 &  0 &  0 &  0 &  0 & -1 &  0 \\ 
														0 &  0 &  0 &  1 &  0 &  0 &  0 &  0 \end{bmatrix}
\]

Now consider $(\phi \after \psi)(\alpha)$ versus $(\psi \after \phi)(\alpha)$:
\begin{align*}
	(\phi \after \psi)(z) &= q_1 + q_6 \alpha - q_3 \alpha^2 - q_8 \alpha^3 - q_5 \imath + q_2 \imath\alpha + q_7 \imath\alpha^2 - q_4 \imath\alpha^3\tag{1}\\
	(\psi \after \phi)(z) &= q_1 - q_6 \alpha - q_3 \alpha^2 + q_8 \alpha^3 - q_5 \imath - q_2 \imath\alpha + q_7 \imath\alpha^2 + q_4 \imath\alpha^3\tag{2}
\end{align*}
Since (1) is not equal to (2), we see that this subgroup is not commutative.

\paragraph*{The automorphisms of $\phi$ and $\psi$ generate a proper subgroup of the group $G$.}
This claim is \emph{false}. Let $H = \langle \phi, \psi \rangle$, the subgroup of $G$ generated by $\phi$ and $\psi$ under composition. Let $a = \phi$ and $b = \phi \after \psi$. Then we have that $a^2 = e$ and $b^2 = e$, where $e$ is the identity map; yet $a b \neq b a$. This describes the dihedral group on 4 elements, $D_8$: so $H = \langle \phi, \psi \rangle = \langle a, b | a^2 = e, b^2 = e, ab \neq ba \rangle = D_8$. Now let $\xi \in G$. Then for any $z \in F$ we have:
\begin{align*}
	\xi(z)	&= \xi(q_1 + q_2 \alpha + q_3 \alpha^2 + q_4 \alpha^3 + q_5 \imath + q_6 \imath\alpha + q_7 \imath\alpha^2 + q_8 \imath\alpha^3)\\
					&= q_1 + q_2 \xi(\alpha) + q_3 \xi(\alpha^2) + q_4 \xi(\alpha^3) + q_5 \xi(\imath) + q_6 \xi(\imath\alpha) + q_7 \xi(\imath\alpha^2) + q_8 \xi(\imath\alpha^3)\\
					&= q_1 + q_2 \xi(\alpha) + q_3 \xi(\alpha)^2 + q_4 \xi(\alpha)^3 + q_5 \xi(\imath) + q_6 \xi(\imath)\xi(\alpha) + q_7 \xi(\imath)\xi(\alpha)^2 + q_8 \xi(\imath)\xi(\alpha)^3
\end{align*}
So the automorphism is determined by where it sends $\alpha$ and $\imath$. The options are:
\begin{align*}
% Real alphas
\alpha \mapsto \alpha, \imath \mapsto \imath 						&= \xi_0 = e\\
\alpha \mapsto \alpha, \imath \mapsto -\imath						&= \xi_1 = a\\
%
\alpha \mapsto -\alpha, \imath \mapsto \imath 					&= \xi_2 = abab = baba\\
\alpha \mapsto -\alpha, \imath \mapsto -\imath					&= \xi_3 = bab\\
% Alpha -> i * alpha
\alpha \mapsto \imath\alpha, \imath \mapsto \imath			&= \xi_4 = ba\\
\alpha \mapsto \imath\alpha, \imath \mapsto -\imath			&= \xi_5 = b\\
%
\alpha \mapsto -\imath\alpha, \imath \mapsto \imath			&= \xi_6 = ab\\
\alpha \mapsto -\imath\alpha, \imath \mapsto -\imath		&= \xi_7 = aba
\end{align*}
Indeed, we cannot send $\imath$ to anything but one of $\pm\imath$ because it has order $2$, whereas $\alpha$ has order $4$; and we cannot send $\alpha$ to anything but one of $\set{\pm\alpha, \pm\imath\alpha}$ because anything else would either (1) reduce to purely an element of $\Q$; or (2) increase the coefficient, for example $\alpha \mapsto \alpha^3$ results in $q_3 \alpha^2 \mapsto q_3 \alpha^6 = 2 q_3 \alpha^2$.

Hence we have enumerated the entire group of automorphisms of $F$, and this is precisely the group generated by $a$ and $b$ (equivalently, $\psi$ and $\phi$).

\paragraph*{There is an injective group homomorphism $G \embed S_4$.}
This claim is \emph{true}. First define a mapping $f: \set{1, 2, 3 ,4} \to F$ by $f(1) = \alpha$, $f(2) = \imath\alpha$, $f(3) = -\alpha$, $f(4) = -\imath\alpha$. Then if $* = G \embed S_4$, in cycle notation we have $\phi^* = (2\;4)$ and $\psi^* = (1\;2\;3\;4)$; since $\phi$ and $\psi$ generate $G$, we have that $G^* \sbsq S_4$. In particular since $G \approx D_8 \sbs S_4$ we know we have an injection.

\paragraph*{There is a surjective homomorphism $G \embed S_4$.}
This claim is \emph{false}. For example, the permutation $(1\;2)$ cannot represent any element of $G$, as such an element would interchange $\alpha$ and $\imath\alpha$ yet fix $-\alpha$ and $-\imath\alpha$, which is incompatible with their action on the base field $F$.

\subsubsection*{$\star$4. Which of the following are true?}
\paragraph*{a. $\C$ is an algebraic closure of $\Q$.}
This is \emph{false}: $\C$ is bigger than the algebraic closure of $\Q$. Since $\Q$ is countable, we expect $\bar{\Q}$ to also be countable. However, $\C$ is uncountable.

\paragraph*{b. $\C$ is an algebraic closure of $\R$.}
This is \emph{true}.

\paragraph*{c. Given an algebraic extension $F/\Q$, there is a \emph{unique} homomorphism $F \to \C$ of fields.}
This is \emph{false}: For example, consider $F = \Q(\sqrt{2})$. Then conjugation (i.e., $\psi(a + b \sqrt{2}) = a - b \sqrt{2}$) is an automorphism of $F$, so if we assume that $\phi: F \to \C$ is a homomorphism, then $\phi \after \psi: F \to \C$ is also a homomorphism. Therefore $\phi$ is not unique.

\paragraph*{d. Given an algebraic extension $F/\Q$, there is a homomorphism (not necessarily unique) $F \to \C$ of fields.}
This is \emph{true}.

\paragraph*{e. There is an algebraically closed field of characteristic 2.}
This is \emph{false}: Consider $f \in \F_2[x]$ where $f(x) = 1 + x + x^2$. Then $f(0) = 1$ and $f(1) = 1 + 1 + 1 = 1 + 0 = 1$, so $f$ has a root that is not in $\F_2$.

\subsubsection*{5. Let $F$ be a stem field for the irreducible polynomial $x^6 - 2 \in \Q[x]$. How many homomorphisms of fields $F \to \R$ are there?}
Let $\alpha = \sqrt[6]{2}$ and $j = e^{\imath\pi / 3}$, a primitive $6^{th}$ root of unity. Then the roots of $x^6 - 2$ are $r_1 = \alpha, r_2 = j\alpha, r_3 = j^2\alpha, r_4 = j^3\alpha = -\alpha, r_5 = j^4\alpha = -j\alpha, r_6 = j^5\alpha = -j^2\alpha$. Now for any stem field $F = \Q(r)$ where $r$ is some root of $x^6 - 2$, we can define a homomorphism $f_n: F \to \R$ by $f_n(r) = r_n$ if $r_n \in \R$ ($f_n |_\Q = \mathrm{id}_\Q$). We see immediately that only $r_1 = \alpha$ and $r_4 = -\alpha$ are real numbers, so there are only \emph{two} field homomorphisms from $F$ into $\R$.

\subsubsection*{6. Let F be a stem field for the irreducible polynomial $x^6 - 2 \in \Q[x]$. How many homomorphisms of fields $F \to \C$ are there?}
As above, but since every root of $x^6 - 2$ is in $\C$ (and they are all distinct!) we have \emph{six} field homomorphisms from $\F$ into $\C$.