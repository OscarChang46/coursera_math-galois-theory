\subsection[Quiz]{Week 4 Quiz}
\subsubsection*{1. Which of the following modules are non-zero?}
\paragraph*{a. $\Z / 3\Z \tensor{\Z} \Z / 5\Z$?}
We use Proposition 11 with $I = (3)$ as our ideal. Hence \[\Z / 3\Z \tensor{\Z} \Z / 5\Z \iso (\Z / 5\Z) / ((3) \cdot (\Z / 5\Z))\] But $(3) \cdot (\Z / 5\Z) = \Z / 5\Z$ since $\gcd(3, 5) = 1$. Therefore the tensor product is zero.

\paragraph*{b. $\Z / 3\Z \tensor{\Z} \Z / 9\Z$?}
Again Prop. 11 yields $\Z / 3\Z \tensor{\Z} \Z / 9\Z \iso (\Z / 9\Z) / ((3) \cdot (\Z / 9\Z))$, but this time $(3) \cdot (\Z / 9\Z) \iso \Z / 3\Z$, so the tensor product is isomorphic to $(\Z / 9\Z) / (\Z / 3\Z) \iso \Z / 3\Z$.

\paragraph*{c. $\Q[x] / (x - 1) \tensor{\Q} \Q[x] / (x + 1)$?}
Note that we cannot use Prop. 11 here, since $(x - 1)$ is not an ideal in $\Q$. 

\paragraph*{d. $\Q[x] / (x - 1) \tensor{\Q[x]} \Q[x] / (x + 1)$?}
Here we \emph{can} use Prop. 11, as $(x - 1)$ is an ideal in $\Q[x]$. Hence \[\Q[x] / (x - 1) \tensor{\Q[x]} \Q[x] / (x + 1) \iso  (\Q[x] / (x + 1)) / ((x - 1) \cdot (\Q[x] / (x + 1)))\] But $x - 1$ and $x + 1$ are relatively prime, since \[\left(\frac{1}{2}\right)(x + 1) + \left(\frac{-1}{2}\right)(x - 1) = \frac{x}{2} + \frac{1}{2} - \frac{x}{2} + \frac{1}{2} = 1\] Therefore $(x - 1) \cdot (\Q[x] / (x + 1)) \iso (\Q[x] / (x + 1))$ and so the tensor product is zero.

\subsubsection*{2. Consider a commutative ring $A$ and an unknown element $x \in A$. Which of the following systems of congruences has a solution for any choice of $a, b \in A$? (Hint: Chinese remainder theorem.)}
We want the two moduli to be relatively prime.


\subsubsection*{3. Which of the following algebras are products of fields (maybe with only one factor)?}


\subsubsection*{4. Which of the following statements are true?}

\subsubsection*{5. Let $k$ be a field, $A$ a $k$-algebra of dimension $2$. Choose $x \in A$, $x \notin k \sbs A$. Then $A$ is generated by $1$ and $x$ and so there is an isomorphism of algebras $k[x]/(x^2 + ax + b) \iso A$ for some $a, b \in k$. Which of the following statements are true?}

Let $L = k[x] / (x^2 + ax + b)$. Since $[A : k] = 2$, we need $[L : k] = 2$, that is, we need the polynomial $x^2 + ax + b$ to have two distinct roots in $k$.

\paragraph*{Case 5a: $k = \Q$.}
\begin{proof}[Claim: There are \emph{infinitely many} non-isomorphic possibilities for $A$.]

\end{proof}

\paragraph*{Case 5b: $k = \F_p$.}
\begin{proof}[Claim: There are \emph{xxx} possibilities for $A$, up to isomorphism.]

\end{proof}

\paragraph*{Case 5c: $k$ is algebraically closed.}
\begin{proof}[Claim: There are \emph{two} possibilities for $A$, up to isomorphism.]

\end{proof}
