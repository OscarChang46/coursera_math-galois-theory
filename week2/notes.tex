\section[Stem field, splitting field, algebraic closure]{Lecture Notes: 29 Feb -- 06 Mar}

\subsection{Stem field. Some irreducibility criteria.}
\subsubsection{Definition, existence, and uniqueness of stem fields.}
\begin{defn}
Let $P \in K[x]$ be irreducible and monic. A \term{stem field} for $P$ is an extension $E \supset K$ such that $E$ contains a root $\alpha$ of $P$ and $E = k[\alpha]$.
\end{defn}

We know that such a thing exists; take, for example $K[x]/(P)$. This is a field, since $P$ is irreducible. On the other hand, \emph{any} stem field $E$ is isomorphic to $K[x]/(P)$. It's easier to define this isomorphism the other way: $K[x]/(P) \to E$ where $f \mapsto f(\alpha)$.

To summarize, we have the following proposition.

\begin{prop}
A stem field exists, and if $E$ and $E^\prime$ are two stem fields for $P \in K[x]$ generated by roots $\alpha$ and $\alpha^\prime$ respectively, then there exists a \emph{unique} isomorphism of $K$-algebras $E \to E^\prime$ taking $\alpha \mapsto \alpha^\prime$.
\end{prop}
\begin{proof}
We've already shown existence. Proving the uniqueness of the isomorphism is also easy. We know that any isomorphism of $K[\alpha]$ with $E^\prime$ is defined by the value it takes on $\alpha$. Now, we have $\phi: K[x]/(P) \to E$ and $\psi: K[x]/(P) \to E^\prime$, so take $\psi^{-1} \after \phi: E \to E^\prime$. This is an isomorphism, and since $\phi$ maps $x \mapsto \alpha$ and $\psi$ maps $x \mapsto \alpha^\prime$ we see that $\psi^{-1} \after \phi$ maps $\alpha \mapsto \alpha^\prime$.
\end{proof}

\begin{rmk}
In particular, if a stem field contains two roots of $P$, there exists a \emph{unique} automorphism taking one root to the other root.
\end{rmk}
\begin{rmk}
If $E$ is a stem field, then $[E:K] = \deg P$; conversely, if $[E:K] = \deg P$ and $E$ contains a root of $P$, then $E$ is a stem field. (Otherwise, its degree over $K$ would be strictly greater than the degree of $P$.)
\end{rmk}

\subsubsection{More criteria for irreducibility.}
\begin{cor}
A polynomial $P \in K[x]$ is irreducible over $K$ if and only if it does not have roots in extensions $L \supset K$ of degree less than or equal to $n / 2$, where $n = \deg P$.
\end{cor}
\begin{proof}
Suppose $P$ is not irreducible. Then it has a prime factor $Q$ such that $\deg Q \leq n/2$, so we can take $L$ as the stem field of $Q$.

Conversely, if $P$ has a root $\alpha \in L$, then $P_{\min}(\alpha, K)$ divides $P$. Then $P$ cannot be irreducible.
\end{proof}

\begin{cor}
Let $P \in K[x]$ be irreducible of degree $n$, and let $L$ be an extension of degree $m$. If $\gcd(n, m) = 1$ then $P$ is irreducible over $L$.
\end{cor}
\begin{proof}
Suppose $Q$ divides $P$ in $L[x]$. Let $M$ be a stem field of $Q$ over $L$. We now have $K \sbs L \sbs M = L[\alpha]$. Then $K(\alpha)$ is a stem field of $P$ over $K$, so $[K(\alpha):K] = n$. On the other hand, if $\deg Q = d$, then $[M:L] = d$ since $M$ is a stem field of $Q$ over $L$. Thus, the total degree $[M:K] = [M:L][L:K] = m d$. But $K(\alpha) \sbsq M$, so $n$ must divide $md$. If $\gcd(n,m) = 1$ then $n$ must divide $d$, but $d$ doesn't exceed $n$ and so $n = d$. Therefore $P$ is irreducible over $L$.
\end{proof}

\subsection{Splitting field.}
Now let $P \in K[x]$ (not necessarily irreducible).

\begin{defn}
A field $L \sps K$ is a \term{splitting field} of $P$ over $K$ if it is an extension where $P$ is \term{split} (i.e., is a product of linear factors) and if it is generated by the roots of $P$. (So it is the smallest field in which $P$ splits.)
\end{defn}

\begin{thm}
(1) A splitting field exists, and its degree over $K$ is less than or equal to $d!$, where $d = \deg P$; and (2) If $L$ and $L^\prime$ are two splitting fields for $P$, then there exists an isomorphism of $K$-algebras $L \to L^\prime$ (but this isomorphism is not necessarily unique).
\end{thm}
\begin{proof}
We will proceed by induction on $d$, the degree of the polynomial $P \in K[x]$.

First, if $d = 1$, then everything is trivial (the splitting field is just $K$ itself).

Now suppose that $d > 1$ and assume the theorem has been proved for all polynomials of degree less than $d$ over any field $K$. In this case, let $Q$ be an irreducible factor of $P$ and let $\alpha$ be a root of $Q$. Then $L_1 = K[\alpha]$ is a stem field of $Q$; over $L_1$, we have $P = (x - \alpha) R$.

By hypothesis, we know that there exists a splitting field $L$ of $R$ over $L_1$ and that $[L:L_1] \leq (\deg R)! \leq (d - 1)!$ since $\deg R \leq d - 1$. This will be a splitting field of $P$ over $k$, and $[L:K] = [L:L_1] [L_1:K] \leq (d-1)! d = d!$.

It remains to prove uniquness up to isomorphism. Let $L$ and $M$ be two splitting fields. Let $\beta$ be a root of $Q$ in $M$, where $Q$ is some irreducible factor of $P$. Then $K[\alpha]$ and $K[\beta]$ are both stem fields for $Q$, and we have an isomorphism $\phi: K[\alpha] \to K[\beta]$ that sends $\alpha$ to $\beta$. Now $P = (x - \beta) S$ in $M[x]$, where $S = \phi(R)$. $M$ is a splitting field of $S$ over $K[\beta]$.

But $M$ is also a $K[\alpha]$-algebra via $\phi$. As such, it is a splitting field of $R$ over $K[\alpha]$. By induction, we have a $K[\alpha]$-isomorphism from $L$ to $M$, so also we have a $K$-isomorphism between $L$ and $M$.
\end{proof}

\begin{rmk}
This isomorphism is \emph{not unique}! In particular, a splitting field can have many $K$-automorphisms. This is in fact the subject of Galois theory: the study of the group of automorphisms.
\end{rmk}

\subsection{An example. Algebraic closure.}

Consider the polynomial $x^3 - 2$ over $\Q$. Its roots are $\sqrt[3]{2}$, $j \sqrt[3]{2}$, and $j^2 \sqrt[3]{2}$, where $j = e^{2\pi\imath / 3}$ (i.e. the primitive third root of unity). The splitting field is then $L = \Q(\sqrt[3]{2}, j)$. Let us now find the automorphisms of $L$.

We have two towers:

\begin{figure}[h]
\centering
\begin{tikzpicture}
	 \matrix (m) [matrix of math nodes, row sep=3em, column sep=3em, minimum width=2em]
			{    & \Q(j)           &                    \\
			  \Q &                 & \Q(\sqrt[3]{2}, j) \\
				   & \Q(\sqrt[3]{2}) &                    \\ };
	 \path[-stealth]
			(m-2-1) edge node [above] {2} (m-1-2)
						  edge node [below] {3} (m-3-2)
			(m-1-2) edge node [above] {\textbf{3}} (m-2-3)
			(m-3-2) edge node [below] {2} (m-2-3);
\end{tikzpicture}
\end{figure}
The minimum polynomial of $j$ over $\Q$ is $x^2 + x + 1$, and indeed this is also the minimum polynomial over $\Q(\sqrt[3]{2})$ (since, e.g., $j$ is not a real number, so it cannot be in that field). Moreover, we conclude that $[\Q(j, \sqrt[3]{2}):\Q(j)]$ must be 3, since the total degree is 6, the degree of $\Q(j)$ over $\Q$ is 2, and degrees multiply in towers.

There must exist a $\Q(j)$-automorphism of $L$, call it $\sigma$, taking $\sqrt[3]{2}$ to $j \sqrt[3]{2}$, because $L$ is a stem field of $x^3 - 2$ over $\Q(j)$, so there are automorphisms that interchange roots. 

There is also a $\Q(\sqrt[3]{2})$-automorphism of $L$, call it $\tau$, taking $j$ to $j^2$, since these are two roots of the same minimal polynomial, and $L$ is a stem field of $x^2 + x + 1$ over $\Q(\sqrt[3]{2})$.

Thus we have a group of automorphisms, $\mr{Aut}(L/K)$, embedded in $S_3$, the group of permutations on 3 elements. In fact, $\mr{Aut}(L/K) = S_3$, since $\sigma$ is a cyclic permutation of roots ($\sqrt[3]{2} \to j^2 \sqrt[3]{2} \to \sqrt[3]{2}$) and $\tau$ is a transposition that fixes $\sqrt[3]{2}$ and exchanges the other roots ($j \sqrt[3]{2} \leftrightarrow j^2 \sqrt[3]{2}$). Together they generate $S_3$.

\subsubsection{Algebraic closure.}
\begin{defn}
A field $K$ is \term{algebraically closed} if any non-constant polynomial has a root in $K$. That is, any non-constant polynomial splits in $K[x]$.
\end{defn}

\begin{ex}
The field of complex numbers, $\C$, has this property (to be proved later, by ``almost'' pure algebra).
\end{ex}

\begin{defn}
An \term{algebraic closure} of $K$ is a field $L$ which is algebraically closed and also algebraic over $K$.
\end{defn}

\begin{thm}
Any field $K$ has an algebraic closure.
\end{thm}
Note that at this point we're saying nothing about \emph{uniqueness}.
\begin{proof}
First, we construct $K_1$ such that for any polynomial $P \in K[x]$ has a root in $K_1$. This is not yet a victory, because we don't know if any polynomial in $K_1[x]$ has a root in $K_1$ itself. So we construct $K_2$ such that any polyomial in $K_1[x]$ has a root in $K_2$, and so forth. We then have $K \sbs K_1 \sbs K_2 \dotsb \sbs K_n \sbs \dotsb$, and take $\bar{K} = \bigcup K_n$. We now claim that $\bar{K}$ is algebraically closed; indeed, any polynomial $P \in \bar{K}[x]$ really has its coefficients in some ``floor'' of this tower. So there exists some $n$ such that $P \in K_n[x]$, which implies that $P$ has a root in $K_{n+1}$, and therefore it has a root in $\bar{K}$. If we learn how to construct these $K_1$, $K_2$, and so on, we will have solved the problem.

We proceed with the construction of $K_1$. Let $S$ be the set of all irreducible elements of $K[x]$, and $A = K[(x_p)_{p \in S}]$, that is, one variable $x_p$ for every $p \in S$. 
($A$ is a very big polynomial ring!)

Let $\mf{i} \sbs A$ be the ideal generated by all $P(x_p)$, $p \in S$. 
We claim that $\mf{i}$ is a proper ideal. 
Indeed, if not, then we can write $1 = \sum_{i=1}^{n} \lambda_i P_i(x_{p_i})$, with the coefficients $\lambda_i \in A$. 
The main point here is that this sum is \emph{finite}. 

Next, take $L$ the splitting field of $\prod_{i=1}^{n} P_i$ over $K$, and let $\alpha_i$ be a root of $P_i$ in $K$. 
Since $A$ is a polynomial ring, it's easy to produce a homomorphism from a polynomial algebra to some other algebra (just note where the mapping sends the variables). 
Hence there exists a homomorphism $\phi: A \to L$ sending $x_{p_i} \mapsto \alpha_i$, and all other $x_p$ to $0$ (that is, when $p \neq p_i$). 
Now we have that $\phi(1) = 0$, since $\phi(P_i(x_{p_i})) = P_i(\alpha_i) = 0$; this is a contradiction, since we need $\phi(1) = 1$.

Having shown that $\mf{i}$ is a proper ideal, we use the fact that any such ideal in a commutative associative ring with unity is contained in a maximal ideal $\mf{m}$ and $A / \mf{m}$ is a field. Take $K_1 = A / \mf{m}$, and continue in the same way to construct $K_2, K_3, \dotsc, K_n, \dotsc$.
\end{proof}
\begin{rmk}[Ideals in a ring]
Any proper ideal (in a commutative associative ring with unity) is contained in a maximal ideal. This is a consequence of Zorn's lemma.
\end{rmk}
\begin{lem}[Zorn]
For $\mc{P}$ a partially ordered set, we say that a subset $\mc{C} \sbs \mc{P}$ is a \term{chain} if, for all $\alpha, \beta \in \mc{C}$, we have $\alpha \leq \beta$ or $\beta \leq \alpha$, where $\leq$ is the order relation on $\mc{P}$. If any non-empty chain in a non-empty $\mc{P}$ has an upper bound, then $\mc{P}$ has maximal elements.
\end{lem}
We will not prove Zorn's lemma, since it's equivalent to the Axiom of Choice, or a meta-theorem, etc.: Relevant to set theory and mathematical foundations, but not to Galois theory.

Now, in this case we have $\mc{P}$ as the set of all proper ideals in $A$ containing $\mf{i}$. We know $\mc{P}$ is non-empty because it contains $\mf{i}$. Any chain $\set{\mf{i}_\alpha}_{\alpha \in J}$ has an upper bound: this is $\bigcup_{\alpha \in J} \mf{i}_\alpha$. (It's easy to check that this is an ideal.) So by Zorn's lemma we know that $\mc{P}$ has maximal elements. Then $\mf{i} \sbs \mf{m}$ a maximal idea. Then $A / \mf{m}$ is a field. Otherwise, some $a \in A / \mf{m}$ would generate a proper ideal, and its pre-image under $\pi: A \to A / \mf{m}$ would strictly contain $\mf{m}$.

\subsection{Extension of homomorphisms. Uniqueness of algebraic closure.}

To sum up, we have just proved the existence of an algebraic closure $\bar{K} = \bigcup_{i = 1}^{\infty} K_i$ where $K_1 \sbs \dotsb \sbs K_i \sbs K_{i+1} \sbs \dotsb$. Each $K_i$ is a field where each $P \in K_{i-1}[x]$ has a root, and was constructed as the quotient of a huge polynomial ring over $K_{i-1}$ by a suitable maximal ideal (first finding a proper ideal, then leveraging Zorn's lemma to find the maximal ideal).

Naturally we might ask if there is a uniqueness result for the algebraic closure. In fact there is, but we need another theorem first.

\begin{thm}[Extension of homomorphisms]
Let $K \sbs L \sbs M$ be algebraic extensions, and embed $K \embed \Omega$ into some algebraic closure (of $K$). Then any homomorphism $\phi: L \to \Omega$ extends to a homomorphism $\tilde{\phi}: M \to \Omega$.
\end{thm}
\begin{proof}
We again apply Zorn's lemma, this time to the following set: \[\mc{E} = \set{(N, \psi) : L \sbs N \sbs M, \text{ and } \psi|_L = \phi}\] 
We know $\mc{E}$ is non-empty because it contains $(L, \phi)$. We equip $\mc{E}$ with a partial order by the following relation: $(N, \psi) \leq (N^\prime, \psi^\prime)$ if $N \sbsq N^\prime$ and $\psi^\prime|_N = \psi$ (that is, $\psi^\prime$ extends $\psi$).

Now, any chain $(N_\alpha, \psi_\alpha)$ has an upper bound $(N, \psi)$; we know that the union $N = \bigcup_{\alpha} N_\alpha$ is a field and a subextension of $M$. 
Then $\psi$ is defined in the obvious way (for $x \in N_\alpha$, set $\phi(x) = \phi_\alpha(x)$).
Hence, by Zorn's lemma we know that $\mc{E}$ has maximal elements.
Let $(N_0, \psi_0)$ be one such element and suppose $N_0 \neq M$, that is, the inclusion is strict.
To obtain a contradiction, take $x \in M / N_0$ and consider $P_{\min}(x, N_0)$.
Let $\alpha \in \Omega$, and define a map $N_0(x) \to \Omega$ by $x \mapsto \alpha$, and equal to $\psi_0$ on $N_0$. 
This contradicts the maximality of the extension $\psi_0$.
Therefore $N_0 = M$, and we take $\tilde{\phi} = \psi_0$.
\end{proof}

\begin{cor}
If $\Omega$, $\Omega^\prime$ are two algebraic closures of $K$, then the are isomorphic as $K$-algebras.
\end{cor}
\begin{proof}[Proof (Sketch)]
Since we have embeddings $i: K \embed \Omega$ and $i^\prime: K \embed \Omega^\prime$, we can extend $i$ (respectively $j$) to a map $\phi: \Omega^\prime \to \Omega$ (respectively, a map $\phi^\prime: \Omega \to \Omega^\prime$). Combining the two gives an isomorphism between $\Omega$ and $\Omega^\prime$.
\end{proof}